\documentclass{article}

\usepackage[english]{babel}
\usepackage{fancyhdr}
\pagestyle{fancy}
\fancyhf{}
\lhead{WINTER 2021 --- MATH 340 HW5}
\rhead{Helen (Yeu) Chen}
\setlength{\parindent}{0cm}
\usepackage{makecell}
\usepackage{amsfonts}
\usepackage{longtable}
\usepackage{amsmath}
\usepackage{amssymb}
\usepackage{amsthm}
\usepackage{systeme}

\fancyfoot[C]{\thepage}

\begin{document}

$ \bullet$ \textbf{Problem 1}
\medskip

\begin{itshape}
Find a basis for $\mathcal{L}(\mathbb{R}^2, \mathbb{R}^3)$, the space of all linear transformations from $\mathbb{R}^2$ to $\mathbb{R}^3$.
\end{itshape}
\medskip

\begin{proof}
$ $\newline
\end{proof}

\newpage
$ \bullet$ \textbf{Problem 2}
\medskip

\begin{itshape}
Let $\mathcal{V}$ and $\mathcal{W}$ be two vector spaces. Recall $\mathcal{L}(\mathcal{V}, \mathcal{W})$ is the set containing all linear transformations from $\mathcal{V}$ to $\mathcal{W}$. 

Let $\vec{v} \ne \vec{0}$ $\vec{v} \in \mathcal{V}$, and $S=\{ +: \mathcal{V} \to \mathcal{W} \; | \; +(\vec{v}) = 0 \}$.

Prove that $\mathcal{S}$ is a subsapce of $\mathcal{L}(\mathcal{V}, \mathcal{W})$. If $dim (\mathcal{V}) =n$ and $dim(\mathcal{W})=m$, what is $dim(\mathcal{S})$?
\end{itshape}
\medskip

\begin{proof}
$ $\newline
\end{proof}

\newpage
$ \bullet$ \textbf{Problem 3}
\medskip

\begin{itshape}
Is $T: M_{2 \times 2}(\mathbb{R}) \to \mathcal{P}_2(\mathbb{R})$ defined by $$T( \begin{pmatrix} a & b \\ c & d \end{pmatrix} ) = a+2bx + (c+d)x^2$$ invertable? If so find its inverse $T^{-1}$.
\end{itshape}
\medskip

\begin{proof}
$ $\newline
\end{proof}

\newpage
$ \bullet$ \textbf{Problem 4}
\medskip

\begin{itshape}
Is $T: M_{2 \times 2} \to M_{2 \times 2}$ defined by $$T(\begin{pmatrix} a & b \\ c & d \end{pmatrix}) = \begin{pmatrix} a+b & a \\ c & c+d \end{pmatrix}$$ invertible? If so find its inverse $T_{-1}$.
\end{itshape}
\medskip

\begin{proof}
$ $\newline
\end{proof}

\newpage
$ \bullet$ \textbf{Problem 5}
\medskip

\begin{itshape}
Find the equation (in the xy plane) of the curve get by rotating the parabola $y=x^2$ counterclockwise of an angle of $\frac{\pi}{6}$ degrees.
\end{itshape}
\medskip

\begin{proof}
$ $\newline
\end{proof}

\newpage
$ \bullet$ \textbf{Problem 6}
\medskip

\begin{itshape}
Give an example of a linear transformation $T: \mathcal{V} \to \mathcal{V}$ that is one to one but not invertable.
\end{itshape}
\medskip

\begin{proof}
$ $\newline
\end{proof}

\newpage
$ \bullet$ \textbf{Problem 7}
\medskip

\begin{itshape}
Given an example of a linear transformation $T: \mathcal{V} \to \mathcal{V}$ that is onto but not invertaible.
\end{itshape}
\medskip

\begin{proof}
$ $\newline
\end{proof}

\newpage
$ \bullet$ \textbf{Problem 8}
\medskip

\begin{itshape}
$\mathcal{B}_1 = \{ 1, x ,x^2 \}$ and $\mathcal{B}_2 = \{ 2x^2-x, 3x^2+1, x^2 \}$ are two bases for $\mathcal{P}_2(\mathbb{R})$. Find the matrices $I_{\mathcal{B}_1}^{\mathcal{B}_2}$, of change of basis from $\mathcal{B}_1$ to $\mathcal{B}_2$ and $I_{\mathcal{B}_2}^{\mathcal{B}_1}$ of change of basis from $\mathcal{B}_2$ to $\mathcal{B}_1$.
\end{itshape}

\end{document}