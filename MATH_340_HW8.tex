\documentclass{article}

\usepackage[english]{babel}
\usepackage{fancyhdr}
\pagestyle{fancy}
\fancyhf{}
\lhead{WINTER 2021 --- MATH 340 HW8}
\rhead{Helen (Yeu) Chen}
\setlength{\parindent}{0cm}
\usepackage{makecell}
\usepackage{amsfonts}
\usepackage{longtable}
\usepackage{amsmath}
\usepackage{amssymb}
\usepackage{amsthm}
\usepackage{systeme}

\fancyfoot[C]{\thepage}

\begin{document}

$ \bullet$ \textbf{Problem 1}
\medskip

\begin{itshape}
Give an example of a vector $\vec{v} \in \mathbb{R}^3$ which is a generalized eigenvector for $A= \begin{pmatrix} 2& 1 & 0 \\ 0&2&0 \\ 0&0&1 \end{pmatrix}$ and $\lambda = 2$, but it is not an eigenvector for $A$.
\end{itshape}
\medskip

\begin{proof}
$ $\newline
I claim that the vector $\vec{v} = \begin{pmatrix} 0 \\1\\0 \end{pmatrix}$ will work.

First show that $\vec{v}$ is an generalized eigenvector with eigenvalue 2, namely, showing $\vec{v} \in \mathcal{N}(A-2I)^3$.
\begin{align*}
(A-2I)^3 &= (\begin{pmatrix} 2& 1 & 0 \\ 0&2&0 \\ 0&0&1 \end{pmatrix} - \begin{pmatrix} 2&0&0 \\ 0&2&0 \\ 0&0&2 \end{pmatrix} )^3\\
&= \begin{pmatrix} 0&1&0 \\0&0&0\\0&0&-1 \end{pmatrix}^3 \\
&= \begin{pmatrix} 0&1&0 \\0&0&0\\0&0&-1 \end{pmatrix} \cdot  \begin{pmatrix} 0&0&0 \\0&0&0\\0&0&1 \end{pmatrix} \\
&=  \begin{pmatrix} 0&0&0 \\0&0&0\\0&0&-1 \end{pmatrix}
\end{align*}
Then we have
\begin{align*}
(A-2I)^3 \vec{v} &=  \begin{pmatrix} 0&0&0 \\0&0&0\\0&0&-1 \end{pmatrix} \begin{pmatrix} 0\\1\\0 \end{pmatrix} \\
&= \begin{pmatrix}0\\0\\0 \end{pmatrix}
\end{align*}
This shows that $\vec{v} = \begin{pmatrix} 0\\1\\0 \end{pmatrix} \in \mathcal{N}(A-2I)^3$ and hence it is a generalized eigenvector.

Next we will show that $\vec{v}$ is not an eigenvector for $A$.
\begin{align*}
A\vec{v} &= \begin{pmatrix} 2& 1 & 0 \\ 0&2&0 \\ 0&0&1 \end{pmatrix} \cdot \begin{pmatrix} 0\\1\\0 \end{pmatrix} \\
&= \begin{pmatrix}1 \\2\\0 \end{pmatrix} \\
&\ne c \begin{pmatrix}0\\1\\0 \end{pmatrix}
\end{align*}
So, $A\vec{v}$ is not a scalar multiple of $\vec{v} = \begin{pmatrix}0\\1\\0 \end{pmatrix}$, hence $\vec{v}$ is not an eigenvector for $A$.
\end{proof}

\newpage
$ \bullet$ \textbf{Problem 2}
\medskip

\begin{itshape}
Suppose $\mathcal{V}$ is an $n$ dimensional complex space, $T \in \mathcal{L}(\mathcal{V})$ and $\lambda=0$ is the only eigenvalue of $T$. Prove that $T^n=0$.
\end{itshape}
\medskip

\begin{proof}
$ $\newline
Recall that if $T \in \mathcal{L}(\mathcal{V})$, $\mathcal{F} = \mathbb{C}$, $dim \; \mathcal{V} = n$ and $\lambda_1, \cdots , \lambda_m$ are distinct eigenvalues of $T$, then $\mathcal{V} = k_{\lambda_1} \oplus \cdots \oplus k_{\lambda_m}$. We can use this theorem here since the $\mathcal{V}$ we have is a finite dimensional complex space.

Since we only have one eigenvalue, $\lambda =0$, we can write $\mathcal{V} = k_0 = \mathcal{N}(T- 0 \cdot I)^n = \mathcal{N}(T^n)$. This is saying that for all $\vec{v} \in \mathcal{V}$, $T^n \vec{v} =0$. In other words, $T^n$ is identically 0, that is, $T^n=0$.
\end{proof}

\newpage
$ \bullet$ \textbf{Problem 3}
\medskip

\begin{itshape}
Prove the Pythagorean theorem: suppose $\vec{u}$ and $\vec{v}$ are orthogonal vectors in an inner product space $\mathcal{V}$. Prove that $|| \vec{u} + \vec{v} ||^2 = || \vec{u} ||^2 +||\vec{v}||^2$.
\end{itshape}
\medskip

\begin{proof}
$ $\newline
Since $\vec{u}$ and $\vec{v}$ are orthogonal, $\langle \vec{u}, \vec{v} \rangle = \langle \vec{v}, \vec{u} \rangle =0$.
\begin{align*}
|| \vec{u} + \vec{v} ||^2 &= \langle  \vec{u} + \vec{v}, \vec{u} + \vec{v} \rangle \\
&= \langle \vec{u} + \vec{v}, \vec{u} \rangle + \langle \vec{u} + \vec{v}, \vec{v} \rangle \\
&= \langle \vec{u} ,\vec{u} \rangle + \langle \vec{v}, \vec{u} \rangle + \langle \vec{u}, \vec{v} \rangle + \langle \vec{v}, \vec{v} \rangle \\
&= || \vec{u} ||^2 +0 + 0 + || \vec{v}||^2 \\
&= ||\vec{u}||^2+ ||\vec{v}||^2
\end{align*}
\end{proof}

\newpage
$ \bullet$ \textbf{Problem 4}
\medskip

\begin{itshape}
Prove $f((x,y), (z,t)) = 2xz+3yt$ is an inner product on $\mathbb{R}^2$. Compute $||(1,2)||$ where $|| \; ||$ is the norm associated with the inner product $f$, that is $||\vec{v}|| = \sqrt{f(\vec{v},\vec{v})}$.
\end{itshape}
\medskip

\begin{proof}
$ $\newline
\textbf{a)}
\begin{align*}
f(\vec{x}+\vec{z},\vec{y}) &= f((x_1,x_2)+ (z_1,z_2), (y_1,y_2)) \\
&= f(x_1+z_1, x_2+z_2),(y_1,y_2)) \\
&= 2[(x_1+z_1)y_1] +3[(x_2+z_2)y_2] \\
&= [2x_1y_1+3x_2y_2] +[2z_1y_1 +3z_2y_2] \\
&= f((x_1,x_2),(y_1,y_2)) + f((z_1,z_2),(y_1,y_2))\\
&= f(\vec{x}, \vec{y}) + f(\vec{z},\vec{y})
\end{align*}

\textbf{b)}
\begin{align*}
f(c\vec{x}, \vec{y}) &= f(c(x_1,x_2), (y_1,y_2) \\
&= f((cx_1,cx_2),(y_1,y_2)) \\
&= 2(cx_1y_1)+3(cx_2y_2) \\
&= c(2x_1y_1+3x_2y_2) \\
&= cf((x_1,x_2),(y_1,y_2)) \\
&= cf(\vec{x}, \vec{y})
\end{align*}

\textbf{c)}
\begin{align*}
f(\vec{x}, \vec{y}) &= f((x_1,x_2),(y_1,y_2)) \\
&= 2x_1y_1 + 3x_2y_2 \\
&= \overline{2x_1y_1 + 3x_2y_2} \text{ (since $x_i, y_i \in \mathbb{R}$)} \\
&= \overline{2y_1x_1 +3y_2x_2} \\
&= \overline{f((y_1,y_2),(x_1,x_2))} \\
&= \overline{f(\vec{y},\vec{x})}
\end{align*}

\textbf{d)}

Given that $\vec{x} =(x_1,x_2) \ne 0$, then $x_1, x_2 \ne 0$.
\begin{align*}
f(\vec{x},\vec{x}) &= f((x_1,x_2),(x_1,x_2)) \\
&= 2x_1x_1 + 3x_2x_2 \\
&= 2x_1^2 +3 x_2^2 
\end{align*}
Since the square is always positive, $2x_1^2 +3 x_2^2 \ge 0$. However, since $x_1, x_2 \ne 0$, $2x_1^2 +3 x_2^2 \ne 0$, so $$f(\vec{x},\vec{x})=2x_1^2 +3 x_2^2 > 0$$
All of above proved that $f((x,y), (z,t)) = 2xz+3yt$ is an inner product on $\mathbb{R}^2$.
\medskip

Now compute $|| (1,2) ||$.
\begin{align*}
|| (1,2) || &= \sqrt{\langle (1,2) , (1,2) \rangle} \\
&= \sqrt{2(1 \cdot 1) + 3 ( 2 \cdot 2)} \\
&= \sqrt{2+12} \\
&= \sqrt{14}
\end{align*}

\end{proof}

\newpage
$ \bullet$ \textbf{Problem 5}
\medskip

\begin{itshape}
Consider $\mathcal{V} = \mathcal{P}^2(\mathbb{R})$, the set of degree 2 polynomials with real coefficients, with inner product $\langle p,q \rangle = \int_{-1}^{1} pq \,dx$.

Calculate $|| 2x+3||$.

Let $\mathcal{S} = \{ p \in \mathcal{P}^2(\mathbb{R}) \; | \; p'(1) =0 \}$. Find orthonormal basis for $\mathcal{S}$.
\end{itshape}
\medskip

\begin{proof}
$ $\newline
\begin{align*}
||2x+3|| &= \sqrt{\langle 2x+3, 2x+3 \rangle} \\
&= \sqrt{\int_{-1}^{1} (2x+3)(2x+3) \, dx} \\
&= \sqrt{\int_{-1}^{1} 4x^2+12x+9 \, dx} \\
&= \sqrt{(\frac{4}{3}x^3 +6x^2+9x) |^{1}_{-1}} \\
&= \sqrt{(\frac{4}{3}+6+9)-(-\frac{4}{3}+6-9))}  \\
&= \sqrt{\frac{8}{3}+18} \\
&= \sqrt{\frac{62}{3}}
\end{align*}
\bigskip

Now find an orthonormal basis for $\mathcal{S} = \{ p \in \mathcal{P}^2(\mathbb{R}) \; | \; p'(1) =0 \}$.

First, we first analyze the set $\mathcal{S}$. Since $p \in \mathcal{P}^2(\mathbb{R})$ we write $p = a+bx+cx^2$. Then 
$$ p'(1) = (b+2cx)|_{x=1} = b+2c = 0$$
This implies that $c=-\frac{1}{2}b$ and for all $p \in \mathcal{S}$ it is in the form $p = a+b(x-\frac{1}{2}x^2)$ for some constant $a,b$.

I claim that the orthonormal basis for this set is $\mathcal{B} = \{ \frac{1}{\sqrt{2}}, \sqrt{\frac{45}{32}}(\frac{1}{6}+x-\frac{1}{2}x^2) \}$.
\medskip

First, show $Span(\mathcal{B}) = \mathcal{S}$. 
Let $p \in Span(\mathcal{B})$ then we can write 
$$
p = a\frac{1}{2} + b \sqrt{\frac{45}{32}}(\frac{1}{6}+x-\frac{1}{2}x^2) 
$$
for some constant $a,b$.

Taking the derivative and evaluate it at 1, we have 
$$p'(1) = b\sqrt{\frac{45}{32}}(1-x)|_{x=1} = 0$$
Hence $p \in \mathcal{S}$ and $Span(\mathcal{B}) \subseteq \mathcal{S}$.

Now let $p \in \mathcal{S}$, then $p$ is in the form $a+b(x-\frac{1}{2}x^2)$ for some constant $a,b$. Then 
\begin{align*}
p &= a+b(x-\frac{1}{2}x^2) \\
&= a(\frac{\sqrt{2}}{\sqrt{2}}) +b(\frac{\sqrt{45/32}}{\sqrt{45/32}}) (\frac{1}{6}-\frac{1}{6}+x-\frac{1}{2}x^2)\\
&= \sqrt{2}a \frac{1}{\sqrt{2}} -\frac{b}{6} + b\sqrt{\frac{32}{45}}[\sqrt{\frac{45}{32}}(\frac{1}{6}+x-\frac{1}{2}x^2)] \\
&=(\sqrt{2}a - \sqrt{2} \frac{b}{6}) \frac{1}{\sqrt{2}} + b\sqrt{\frac{32}{45}}[\sqrt{\frac{45}{32}}(\frac{1}{6}+x-\frac{1}{2}x^2)]
\end{align*}
This shows $p \in Span(\frac{1}{\sqrt{2}}, \sqrt{\frac{45}{32}}(\frac{1}{6}+x-\frac{1}{2}x^2))$ and $\mathcal{S} \subseteq Span(\mathcal{B})$.

Hence $\mathcal{S} = Span(\mathcal{B})$.
\medskip

We then will prove that $\mathcal{B}$ is an orthonormal set.

Checking $\langle \vec{b}_i, \vec{b}_j \rangle = 0$ if $i \ne j$.
\begin{align*}
& \int^{1}_{-1} \frac{1}{\sqrt{2}} \cdot \sqrt{\frac{45}{32}}(\frac{1}{6}+x-\frac{1}{2}x^2) \, dx \\
=& \sqrt{\frac{45}{64}} \int^{1}_{-1} \frac{1}{6}+x-\frac{1}{2}x^2 \, dx \\
=& \sqrt{\frac{45}{64}} [\frac{1}{6}x+\frac{1}{2} x^2 - \frac{1}{6}x^3]_{-1}^{1} \\
=& \sqrt{\frac{45}{64}}[(\frac{1}{6}+\frac{1}{2}-\frac{1}{6}) - (-\frac{1}{6}+\frac{1}{2}+\frac{1}{6})] \\
=& \sqrt{\frac{45}{64}} (\frac{1}{2}-\frac{1}{2}) \\
=& 0
\end{align*}
Checking $\langle \vec{b}_i, \vec{b}_j \rangle = 1$ if $i = j$.
\begin{align*}
\int_{-1}^{1} \frac{1}{\sqrt{2}} \cdot \frac{1}{\sqrt{2}} \, dx &= \int_{-1}^{1} \frac{1}{2} \, dx \\
&= \frac{1}{2} [x]_{-1}^{1} \\
&= \frac{1}{2} \cdot 2 \\
&= 1
\end{align*}
\begin{align*}
& \int_{-1}^{1}  \sqrt{\frac{45}{32}}(\frac{1}{6}+x-\frac{1}{2}x^2)) \cdot  \sqrt{\frac{45}{32}}(\frac{1}{6}+x-\frac{1}{2}x^2)) \, dx \\
=& \frac{45}{32} \int_{-1}^{1} (\frac{1}{6}+x-\frac{1}{2}x^2)^2 \, dx \\
=& \frac{45}{32} \int_{-1}^{1} \frac{1}{36} + \frac{1}{3}x - \frac{1}{6}x^2 +x^2 -x^3 +\frac{1}{4}x^4 \, dx \\
=& \frac{45}{32} \int_{-1}^{1} \frac{1}{36} + \frac{1}{3}x +\frac{5}{6}x^2  -x^3 +\frac{1}{4}x^4 \, dx \\
=& \frac{45}{32} [\frac{1}{36}x + \frac{1}{6}x^2 +\frac{5}{18}x^3- \frac{1}{4}x^4+\frac{1}{20}x^5]^{1}_{-1} \\
=& \frac{45}{32} [(\frac{1}{36} + \frac{1}{6} +\frac{5}{18}- \frac{1}{4}+\frac{1}{20}) -(-\frac{1}{36} + \frac{1}{6} -\frac{5}{18}- \frac{1}{4}-\frac{1}{20})]^{1}_{-1} \\
=& \frac{45}{32} (\frac{49}{180}+\frac{79}{180}) \\
=&\frac{45}{32} \cdot \frac{32}{45} \\
=& 1
\end{align*}
This shows $\mathcal{B}$ is an orthonormal set. Since it is orthonormal, we have shown in class that it must be independent. So $\mathcal{B}$ spans $\mathcal{S}$ and linearly independent. Hence it is a orthonormal basis for $\mathcal{S}$.


\end{proof}

\newpage
$ \bullet$ \textbf{Problem 6}
\medskip

\begin{itshape}
Use linear algebra to prove that 
$$ (\frac{a_1+a_2+ \cdots + a_n}{n})^2 \le \frac{a^2_{1}+\cdots +a^2_{n}}{n}$$ for all $a_1, \cdots, a_n \in \mathbb{R}$.
\end{itshape}
\medskip

\begin{proof}
$ $\newline
Recall the Cauchy Schwartz Inequality: $| \langle \vec{v}, \vec{w} \rangle | \le || \vec{v} || \cdot ||\vec{w}||$

For any given positive integer $n$, take $\mathcal{V} = \mathbb{R}^n$ with regular inner product, $\vec{v} = (a_1,a_2, \cdots , a_n)$, and $\vec{w} = (1_1,1_2, \cdots, 1_n)$ (the subscripts of 1 are just for us to track how many 1's we have). Then we have
\begin{align*}
|\langle (a_1, \cdots, a_n), (1_1, \cdots, 1_n) \rangle | &\le || (a_1, \cdots, a_n)|| \cdot || (1_1, \cdots, 1_n) || \\
| a_1 + \cdots + a_n | &\le  \sqrt{a_1^2+ \cdots + a_n^2} \cdot \sqrt{1_1 + \cdots + 1_n} \\
| a_1 + \cdots + a_n | &\le  \sqrt{a_1^2+ \cdots + a_n^2} \cdot \sqrt{n} \\
(a_1 + \cdots + a_n )^2 &\le  (a_1^2+ \cdots + a_n^2) \cdot n\\
\frac{(a_1 + \cdots + a_n )^2}{n^2} &\le  \frac{a_1^2+ \cdots + a_n^2}{n} \text{ (since $\frac{1}{n^2}$ is positive)} \\
(\frac{(a_1 + \cdots + a_n )}{n})^2 &\le  \frac{a_1^2+ \cdots + a_n^2}{n}
\end{align*}
\end{proof}
\end{document}