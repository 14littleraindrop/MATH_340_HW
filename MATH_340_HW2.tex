\documentclass{article}

\usepackage[english]{babel}
\usepackage{fancyhdr}
\pagestyle{fancy}
\fancyhf{}
\lhead{WINTER 2021 --- MATH 340 HW2}
\rhead{Helen (Yeu) Chen}
\setlength{\parindent}{0cm}
\usepackage{makecell}
\usepackage{amsfonts}
\usepackage{longtable}
\usepackage{amsmath}
\usepackage{amssymb}
\usepackage{amsthm}
\usepackage{systeme}

\fancyfoot[C]{\thepage}

\begin{document}


$ \bullet$ \textbf{Problem 1}
\medskip

\begin{itshape}
Let $\mathcal{V} = \mathcal{P}(\mathbb{R})$, the space of all polynomials with real coefficients.

Describe $Span \{ 1,x,x^2 \}$.
\end{itshape}
\medskip

\begin{proof}
$ $\newline
Claim: $Span \{ 1,x,x^2 \} = \mathcal{P}_2(\mathbb{R})$.
\medskip

The span of set $\{ 1, x, x^2 \} $ by definition is the set of all the linear combinations of the elements in the set. So, $Span \{ 1,x,x^2 \} = a \cdot 1 + b \cdot x + c \cdot x^2 = a +bx+cx^2$ where $a,b,c$ are elements of the field (so in this case, $\mathbb{R}$). And it is trivial that $a+bx+cx^2$ includes any polynomial which has a degree less than or equal to 2, hence, $Span \{ 1,x,x^2 \} = \mathcal{P}_2(\mathbb{R})$.
\end{proof}

\newpage
$ \bullet$ \textbf{Problem 2}
\medskip

\begin{itshape}
Let $\mathcal{V} = \mathcal{P}(\mathbb{R})$. Is $x^3-3x+5$ in $Span\{ x^3+2x^2-x+1, x^3+3x^2-1 \}$?
\end{itshape}
\medskip

\begin{proof}
$ $\newline
Claim: Yes.

\medskip
We will prove this by writing $x^3-3x+5$ as a linear combination of $x^3+2x^2-x+1$ and $x^3+3x^2-1$. That is,
\begin{align*}
x^3-3x+5 &= a(x^3+2x^2-x+1) + b(x^3+3x^2-1) \\
&= (a+b)x^3+(2a+3b)x^2 -ax +(a-b)
\end{align*}
where $a ,b \in \mathbb{R}$. This gives a system of equation,
\[
\systeme*{a+b=1,2a+3b =0,-a=-3, a-b =5}
\]
The third equation tells us that $a=3$, and plug it back to the first equation we get $b=-2$. It is easy to see that these value also satisfy the remaining two equations. Hence, 
$$ x^3-3x+5 = 3(x^3+2x^2-x+1) -2(x^3+3x^2-1)$$
and so $x^3-3x+5$ is in $Span\{ x^3+2x^2-x+1, x^3+3x^2-1 \}$


\end{proof}

\newpage
$ \bullet$ \textbf{Problem 3}
\medskip

\begin{itshape}
Let $\mathcal{V} = \mathbb{R}^{[0,1]}$, the space of all real functions defined on $[0,1]$. Let $\mathcal{S} = \{ f:[0,1] \to \mathbb{R} | \exists x,f(x)=1 \}$.

Is $\mathcal{S}$ linearly independent?
\end{itshape}
\medskip

\begin{proof}
$ $\newline
Claim: $\mathcal{S}$ is not linearly independent.
\medskip
Consider the following function in $\mathcal{S}$:
$$f_1(x) = 1-x, f_2(x)=x, f_3 = 1$$
Notice that they are indeed in $\mathcal{S}$ since $f_1(0)= f_2(1)= f_3(x)=1$.
Also note that these three functions are not linearly independent since
\begin{align*}
f_1(x) + f_2(x) + (-1) \cdot f_3(x) &= (1-x)+x-1 \\
&= 0
\end{align*} 
Hence, $\mathcal{S}$ is linearly dependent.
\end{proof}

\newpage
$ \bullet$ \textbf{Problem 4}
\medskip

\begin{itshape}
Suppose that $\mathcal{V}$ is a vector space and that $\mathcal{S}_1 \subseteq \mathcal{S}_2 \subseteq \mathcal{V}$. Prove that if $\mathcal{S}_1$ spans $\mathcal{V}$ then $\mathcal{S}_2$ also spans $\mathcal{V}$, and if $\mathcal{S}_2$ is independent then  $\mathcal{S}_1$ is also independent.
\end{itshape}
\medskip

\begin{proof}
$ $\newline
First show $\mathcal{S}_1$ spans $\mathcal{V} \Rightarrow \mathcal{S}_2$ spans $\mathcal{V}$.

Since $\mathcal{S}_1$ spans $\mathcal{V}$, by theorem proved in class, $\mathcal{S}$ contains a basis for $\mathcal{V}$. Call this basis $\mathcal{B}$, then $\mathcal{B} \subseteq \mathcal{S}_1 \subseteq \mathcal{S}_2$. Notice that for all vectors $\vec{v} $ in $\mathcal{S}_2$ which is not in $\mathcal{B}$, since $\vec{v} \in \mathcal{V}$ so we can write $\vec{v}$ as a linear combination of elements in $\mathcal{B}$. Hence, any linear combination of vectors in $\mathcal{S}_2$ is also a linear combination of vectors in $\mathcal{B}$. So $Span(\mathcal{S}_2) =Span(\mathcal{B}) = \mathcal{V}$.

\bigskip
Now show that $\mathcal{S}_2$ independent $\Rightarrow \mathcal{S}_1$ independent.

To show that $\mathcal{S}_1$ in linearly independent, we want to show that for any finite list $\vec{v}_1, \vec{v}_2, \cdots , \vec{v}_n$ of vectors in $\mathcal{S}_1$, $\vec{v}_1, \cdots, \vec{v}_n$ are linearly independent. For any such list, since $\mathcal{S}_1 \subseteq \mathcal{S}_2$, $\vec{v}_1, \cdots, \vec{v}_n \in \mathcal{S}_2$. And because $\mathcal{S}_2$ is a linearly independent set, $\vec{v}_1, \cdots, \vec{v}_n$ must be linearly independent. This shows that $\mathcal{S}_1$ is independent. 

\end{proof}

\newpage
$ \bullet$ \textbf{Problem 5}
\medskip

\begin{itshape}
Let $\mathcal{U}$, $\mathcal{V} \le \mathbb{R}^4$, $\mathcal{U} = Span \{ (1,1,0,0) \}$, $\mathcal{V} = Span \{ (0,0,1,0), (0,0,0,1) \}$.

Verify $\mathcal{U} + \mathcal{V}$ is a direct sum and find a basis for $\mathcal{U} \oplus \mathcal{V}$.
\end{itshape}
\medskip

\begin{proof}
$ $\newline
First show $\mathcal{U} + \mathcal{V}$ is a direct sum by showing that their intersection is the zero vector.
\begin{align*}
\mathcal{U} \cap \mathcal{V} &= Span\{ (1,1,0,0) \} \cap Span\{ (0,0,1,0) , (0,0,0,1) \} \\
&= \{ (x,x,0,0) | x \in \mathbb{R} \} \cap \{ (0,0,y,z) | y,z, \in \mathbb{R} \} \\
&= \{ (0,0,0,0) \} \\
&= \vec{0}
\end{align*}
Hence the sum between $\mathcal{U}$ and $\mathcal{V}$ is direct.
\bigskip

Next, I claim that a basis for $\mathcal{U} \oplus \mathcal{V}$ is $\mathcal{B} = \{ (1,1,0,0), (0,0,1,0), (0,0,0,1) \}$.

Clearly, $\mathcal{B}$ is a linearly independent set, so what is left to show is that $\mathcal{B}$ spans $\mathcal{U} \oplus \mathcal{V}$.

Let $\vec{x} \in \mathcal{U} \oplus \mathcal{V}$, then we can write $\vec{x}$ as $\vec{x} = \vec{u} + \vec{v}$, where $\vec{u} \in \mathcal{U}$ and $\vec{v} \in \mathcal{V}$. Since $\vec{u} \in \mathcal{U} = Span\{ (1,1,0,0) \} $, it can be written as a linearly combination of $(1,1,0,0)$, so $\vec{u} = a(1,1,0,0)$ (with $a \in \mathbb{R}$). Similarly, since $\vec{v} \in \mathcal{V}=Span\{ (0,0,1,0), (0,0,0,1) \}$ we can write it as $\vec{v} = b(0,0,1,0) + c(0,0,0,1)$ (with $b, c \in \mathbb{R}$). Hence, $\vec{x} = \vec{u} +\vec{v} =  a(1,1,0,0) + b(0,0,1,0)+c(0,0,0,1)$ and so $\vec{x} \in Span(\mathcal{B})$. This shows $\mathcal{U} \oplus \mathcal{V} \subseteq Span(\mathcal{B})$.

Now let $\vec{x} \in Span(\mathcal{B})$, then $\vec{x}$ can be written as a linear combination of vectors in $\mathcal{B}$, $\vec{x} = a(1,1,0,0) + b(0,0,1,0)+ c(0,0,0,1)$ (where $a,b,c \in \mathbb{R}$). Notice that $a(1,1,0,0) \in Span\{ (1,1,0,0) \} = \mathcal{U}$ and $b(0,0,1,0) + c(0,0,0,1) \in Span \{ (0,0,1,0), (0,0,0,1) \} = \mathcal{V}$. So, $\vec{x}$ is the sum of vectors from $\mathcal{U}$ and $\mathcal{V}$, hence, $\vec{x} \in \mathcal{U} \oplus \mathcal{V}$ and $Span(\mathcal{B}) \subseteq \mathcal{U} \oplus \mathcal{V}$.

The above shows that $Span(\mathcal{B}) = \mathcal{U} \oplus \mathcal{V}$, where $\mathcal{B} = \{ (1,1,0,0), (0,0,1,0), (0,0, 0,1) \}$.
\end{proof}

\newpage
$ \bullet$ \textbf{Problem 6}
\medskip

\begin{itshape}
Verify that $\mathcal{S} = \{ x+1, x-1 \}$ is independent in $\mathcal{P}_{3} (\mathbb{R})$ and extend $\mathcal{S}$ to a basis for $\mathcal{P}_{3} (\mathbb{R})$.
\end{itshape}
\medskip

\begin{proof}
$ $\newline
First show $\mathcal{S} = \{ x+1 , x-1 \} $ is independent. 

Let $a(x+1) + b(x-1) = 0$, then rearrange gives $(a+b)x + (a-b) =0$. This further gives,
\[
\systeme{a+b=0,a-b=0}
\]
Solving the system equation one will find that $2a=2b=0$, so it is trivial to conclude that $a=b=0$. This shows $\mathcal{S}$ is linearly independent.
\bigskip

By theorem proved in class, we know that since $\mathcal{S}$ is independent, $\mathcal{S} \subseteq \mathcal{P}_3(\mathbb{R})$ can be extended to a basis of $\mathcal{P}_3(\mathbb{R})$. I claim that $\mathcal{B} = \{ x+1, x-1, x^2, x^3 \}$ is a basis of $\mathcal{P}_3(\mathbb{R})$.

Let $\vec{v} \in Span(\mathcal{B})$, then
\begin{align*}
\vec{v} &= a(x+1)+b(x-1)+cx^2+dx^3 \\
&= (a-b)+(a+b)x+cx^2+dx^3
\end{align*}
where $a,b,c,d \in \mathbb{R}$. So $\vec{v} \in \mathcal{P}_3(\mathbb{R})$ and $Span(\mathcal{B}) \subseteq \mathcal{P}_3(\mathbb{R})$.

Let $\vec{v} \in \mathcal{P}_3(\mathbb{R})$, then $\vec{v} = a+bx+cx^2+dx^3$ where $a,b,c,d \in \mathbb{R}$. Take
\[ 
\systeme*{a = a'-b', b = a'+b', c=c', d=d'}
\]
then
\begin{align*}
\vec{v} &= (a'-b') + (a'+b')x + c'x^2+d'x^3 \\
&= a'(x+1)+b'(x-1)+c'x^2+d'x^3
\end{align*}
Hence, $\vec{v}$ is a linear combination of vectors in $\mathcal{B}$, that is, $\vec{v} \in Span(\mathcal{B})$. So, $\mathcal{P}_3(\mathbb{R}) \subseteq Span(\mathcal{B})$.

The above shows $\mathcal{P}_3(\mathbb{R}) = Span(\mathcal{B})$, in other words, $\mathcal{B}$ spans $\mathcal{P}_3(\mathbb{R})$.

Since $x^2$ and $x^3$ are clearly linearly independent to each other and not in $Span(\mathcal{S})$, $\mathcal{B}$ is linearly independent so it is a basis of $\mathcal{P}_3(\mathbb{R})$.
\end{proof}

\newpage
$ \bullet$ \textbf{Problem 7}
\medskip

\begin{itshape}
Let $\mathcal{U} = \{ p \in \mathcal{P}_4(\mathbb{R}) | \int_{-1}^1 p(x) \,dx = 0 \}$

Prove $\mathcal{U}$ is a subspace of $\mathcal{P}_4 (\mathbb{R})$.

Find a basis for $\mathcal{U}$. Extend the basis you found to a basis for $\mathcal{P}_4(\mathbb{R})$.

Find $\mathcal{W} \le \mathcal{P}_4(\mathbb{R})$ such that $\mathbb{P}_4(\mathbb{R}) = \mathcal{U} \oplus \mathcal{W}$.
\end{itshape}
\medskip

\begin{proof}
$ $\newline
Show that $\mathcal{U} \le \mathcal{P}_4(\mathbb{R})$.

1) $\vec{0} = 0 \in \mathcal{U}$:

Clearly integrate zero over $[-1, 1]$ is still zero. Hence, indeed $p(x) = 0 \in \mathcal{U}$.
\medskip

2) closed under addition:

let $f(x), g(x) \in \mathcal{U}$, then $\int_{-1}^1 f(x) \, dx =0$ and $\int_{-1}^1 g(x) \, dx =0$. Since
\begin{align*}
\int_{-1}^1 (f+g)(x) \, dx &= \int_{-1}^1 (f(x) + g(x)) \, dx =0 \\
&= \int_{-1}^1 f(x) \, dx + \int_{-1}^1 g(x) \, dx \\
&= 0 + 0 \\
&=0
\end{align*}
Thus, $(f+g)(x) \in \mathcal{U}$.
\medskip

3) closed under scalar multiplication:

Let $f(x) \in \mathcal{U}$ and $\lambda \in \mathcal{F} = \mathbb{R}$, then $\int_{-1}^1 f(x) \, dx =0$. Since 
\begin{align*}
\int_{-1}^1 \lambda \cdot f(x) \, dx &= \lambda \int_{-1}^1 f(x) \, dx \\
&= \lambda \cdot 0 \\
&= 0 
\end{align*}
So $\lambda f(x) \in \mathcal{U}$
\medskip

This shows that $\mathcal{U}$ is a subspace of $\mathcal{P}_4 (\mathbb{R})$.
\bigskip

I claim that $\mathcal{B} = \{ -\frac{1}{3}+x^2, -\frac{1}{5}+x^4 , x, x^3 \}$ is a basis of $\mathcal{U}$. 
First of all, $\mathcal{B}$ is clearly independent because non of the elements share the same $x^n$ term. What is left to be shown is $Span(\mathcal{B}) = \mathcal{U}$.

Let $\vec{v} \in Span(\mathcal{B})$, then we can write $\vec{v} = a(-\frac{1}{3}+x^2)+b(-\frac{1}{5}+x^4 )+cx+dx^3$ with $a,b,c,d \in \mathbb{R}$. Since 
\begin{align*}
\int_{-1}^1 \vec{v} \, dx &= \int_{-1}^1 a(-\frac{1}{3}+x^2)+b(-\frac{1}{5}+x^4 )+cx+dx^3 \, dx \\
&= \int_{-1}^1 a(-\frac{1}{3}+x^2) \, dx + \int_{-1}^1 b(-\frac{1}{5}+x^4 ) \, dx + \int_{-1}^1 cx \, dx+ \int_{-1}^1 dx^3 \, dx \\
&= a(-\frac{1}{3}x+\frac{1}{3}x^3) \bigm|_{-1}^1 +b(-\frac{1}{5}x+\frac{1}{5}x^5) \bigm|_{-1}^1+ \frac{c}{2}x^2 \bigm|_{-1}^1 + \frac{d}{4}x^4 \bigm|_{-1}^1 \\
&= a[(-\frac{1}{3}+\frac{1}{3}) -(\frac{1}{3}-\frac{1}{3})] + b[(-\frac{1}{5}+\frac{1}{5}) -(\frac{1}{5}-\frac{1}{5})] + \frac{c}{2}(1-1)+ \frac{d}{4}(1-1) \\
&= a \cdot 0 + b\cdot 0 + c \cdot 0 +d \cdot 0 \\
&= 0
\end{align*}
Hence $\vec{v} \in \mathcal{U}$ and $Sapn(\mathcal{B}) \subseteq \mathcal{U}$.

Now let $\vec{v} = a + bx+cx^2+dx^3+ex^4 \in \mathcal{U}$ (where $a,b,c,d$ are some unknown coefficients), then
\begin{align*}
0 &= \int_{-1}^1 \vec{v} \, dx \\
&= \int_{-1}^1 a+bx+cx^2+dx^3+ex^4 \, dx \\
&= (ax + \frac{b}{2}x^2 + \frac{c}{3}x^3 + \frac{d}{4}x^4 + \frac{e}{5}x^5) \bigm|_{-1}^1 \\
&= (a+ \frac{b}{2} + \frac{c}{3} +\frac{d}{4} +\frac{e}{5}) - (-a+ \frac{b}{2} - \frac{c}{3} +\frac{d}{4} -\frac{e}{5}) \\
&= 2(a + \frac{c}{3} +\frac{e}{5})
\end{align*}
Hence $\vec{v} \in \mathcal{U}$ implies that $a+\frac{c}{3} +\frac{e}{5} =0$. Then,
\begin{align*}
\vec{v} &= a + bx+cx^2+dx^3+ex^4 \\
&= -\frac{c}{3} -\frac{e}{5} + bx+cx^2+dx^3+ex^4 \\
&= c(-\frac{1}{3}+x^2) + e(-\frac{1}{5} +x^4) +bx+dx^3
\end{align*}
This shows that $\vec{v}$ is a linear combination of vectors in $\mathcal{B}$, hence $\vec{v} \in Span(\mathcal{B})$ and $\mathcal{U} \subseteq Span(\mathcal{B})$.

Hence, $Span(\mathcal{B}) = \mathcal{U}$ with $\mathcal{B} = \{ -\frac{1}{3}+x^2, -\frac{1}{5}+x^4 , x, x^3 \}$.

Now we want to extend $\mathcal{B}$ to a basis for $\mathcal{P}_4(\mathbb{R})$. I claim that by just adding the constant $1$, we can extend $\mathcal{B}$ in to a basis for $\mathcal{P}_4(\mathbb{R})$. 

By the similar reasoning we used on $\mathcal{B}$, $\mathcal{B'} = \{ -\frac{1}{3}+x^2, -\frac{1}{5}+x^4 , x, x^3 , 1\}$ is clearly independent. Then we will show that $Span(\mathcal{B'}) = \mathcal{P}_4(\mathbb{R})$.

Let $\vec{v} \in Span(\mathcal{B'})$, then 
\begin{align*}
\vec{v} &= a(-\frac{1}{3}+x^2)+b(-\frac{1}{5}+x^4)+ cx+dx^3+e \cdot 1 \\
&= (-a\frac{1}{3} -b\frac{1}{5} +e) + cx +ax^2+dx^3+bx^4
\end{align*}
so $\vec{v} \in \mathcal{P}_4(\mathbb{R})$ and hence $Span(\mathcal{B'}) \subseteq \mathcal{P}_4(\mathbb{R})$.

Let $\vec{v} \in \mathcal{P}_4(\mathbb{R})$, then 
\begin{align*}
\vec{v} &= a+bx+cx^2+dx^3+ex^4 \\
&= \frac{c}{3} -\frac{c}{3} +\frac{e}{5} -\frac{e}{5} + a+bx+cx^2+dx^3+ex^4 \\
&= (\frac{c}{3} + \frac{e}{5} +a) + bx + (-\frac{c}{3} + cx^2) +dx^3+(-\frac{e}{5}+ex^4) \\
&= (\frac{c}{3} + \frac{e}{5} +a) \cdot 1 + b  x + c  (-\frac{1}{3} + x^2) + d  x^3 + e  (-\frac{1}{5}+x^4)
\end{align*}
so $\vec{v}$ is a linear combination of vectors in $\mathcal{B'}$, hence $\mathcal{P}_4(\mathbb{R}) \subseteq \mathcal{B'}$.

Thus, the above shows $Span(\mathcal{B'}) = \mathcal{P}_4(\mathbb{R})$, and so $\mathcal{B'} = \{ -\frac{1}{3}+x^2, -\frac{1}{5}+x^4 , x, x^3 , 1\}$ is a basis for $\mathcal{P}_4(\mathbb{R})$.
\bigskip

Now we wish to find $\mathcal{W} \le \mathcal{P}_4(\mathbb{R})$ such that $\mathcal{P}_4(\mathbb{R}) = \mathcal{U} \oplus \mathcal{W}$.

Claim: $\mathcal{W} = \{a \in \mathbb{R} \}$ works (i.e. all constant functions).

It is trivial that one basis for $\mathcal{W}$ is the constant function one, $f(x)=1$. And since we have concluded that $\mathcal{B'}$ is a basis for $\mathcal{P}_4(\mathbb{R})$. For all $\vec{v} \in \mathcal{P}_4(\mathbb{R})$,
$$
\vec{v} = [a(-\frac{1}{3}+x^2) + b (-\frac{1}{5}+x^4) + cx+dx^3]+[e \cdot 1] 
$$
Notice that the first square bracket is a vector in $\mathcal{U}$ because it is a linear combination of vectors in $\mathcal{B}$ (basis for $\mathcal{U}$); similarly, the last square bracket is a vector in $\mathcal{W}$. This implies that $\vec{v} \in \mathcal{U} + \mathcal{W}$, hence $\mathcal{P}_4(\mathbb{R}) \subseteq \mathcal{U} + \mathcal{W}$. The other direction of inclusion is also true since if $\vec{v} \in \mathcal{U} + \mathcal{W}$ then it can be written as a sum of vectors in $\mathcal{U}$ and $\mathcal{W}$, but each of them can be written as linear combinations of vectors from $\mathcal{B'}$, hence $\vec{v} \in \mathcal{P}_4(\mathbb{R})$. Thus, $\mathcal{P}_4(\mathbb{R}) = \mathcal{U} + \mathcal{W}$.
\medskip

To show this sum is direct, let $\vec{v} \in \mathcal{U} \cap \mathcal{W}$. Since $\vec{v} \in \mathcal{W}$, it must be a constant function. However, the only constant function that integrates to 0 over $[0,1]$ is the zero function, so $\vec{v} = 0$. Thus $\mathcal{U} \cap \mathcal{W} = \{ 0 \}$ and hence the sum is direct.
\end{proof}

\newpage
$ \bullet$ \textbf{Problem 8}
\medskip

\begin{itshape}
Prove $\mathbb{R}^ \infty$ (the space of all sequences of real numbers) is not finite dimensional.
\end{itshape}
\medskip

\begin{proof}
$ $\newline
\end{proof}


\end{document}