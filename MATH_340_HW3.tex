\documentclass{article}

\usepackage[english]{babel}
\usepackage{fancyhdr}
\pagestyle{fancy}
\fancyhf{}
\lhead{WINTER 2021 --- MATH 340 HW3}
\rhead{Helen (Yeu) Chen}
\setlength{\parindent}{0cm}
\usepackage{makecell}
\usepackage{amsfonts}
\usepackage{longtable}
\usepackage{amsmath}
\usepackage{amssymb}
\usepackage{amsthm}
\usepackage{systeme}

\fancyfoot[C]{\thepage}

\begin{document}


$ \bullet$ \textbf{Problem 1}
\medskip

\begin{itshape}
Prove that the function $T: \mathcal{P}_2(\mathbb{R}) \to \mathcal{P}_4(\mathbb{R})$, $T(p(x)) = p(x) \cdot (x^2 +1)$ is a linear transformation. Describe $\mathcal{N}(T)$. Is $x^4$ in $\mathcal{R}(T)$?
\end{itshape}
\medskip

\begin{proof}
$ $\newline
First show that $T$ is a linear transformation.

Let $p(x), q(x) \in \mathcal{P}_2(\mathbb{R})$ and $\lambda \in \mathbb{R}$. Then
\begin{align*}
T(\lambda p(x)) &= \lambda p(x) \cdot (x^2+1) \\
&= \lambda (p(x) \cdot (x^2+1)) \\
&= \lambda T(p(x))
\end{align*}
\begin{align*}
T(p(x) + q(x)) &= (p(x) +q(x)) \cdot (x^2+1) \\
&= (p(x) \cdot (x^2+1)) + (q(x) \cdot (x^2+1)) \\
&= T(p(x)) + T(q(x))
\end{align*}
This shows that $T$ is a linear transformation.
\medskip

Claim: $\mathcal{N}(T) = \{ p(x) = 0 \}$

Write $p(x) \in \mathcal{P}_2(\mathbb{R})$ as $p(x) = ax^2+bx+c$. We want to fine all $p(x)$ such that $T(p(x)) = 0$.
\begin{align*}
T(p(x)) &= T(ax^2+bx+c) \\
&= (ax^2+bx+c) \cdot (x^2+1) \\
&= ax^4+bx^3+(a+c)x^2+bx+c
\end{align*}
So, $T(p(x)) = 0$ if and only if $a,b,c = 0$. Hence, only $p(x) = 0$ is in the null space of $T$.
\medskip

Claim: $x^4 \notin \mathcal{R}(T)$. If $x^4 \in \mathcal{R}(T)$, then there exists a $p(x) = ax^2+bx+c \in \mathcal{P}_2(\mathbb{R})$ such that $T(p(x)) =ax^4+bx^3+(a+c)x^2+bx+c= x^4$. However, the $x^4, x^3$ and constant components imply that $a=1, b=c=0$, but these coefficients will result in a nonzero $x^2$ component.  Hence, there is not $p(x)$ such that $T(p(x)) = x^4$, and so $x^4 \notin \mathcal{R}(T)$.
\end{proof}

\newpage
$ \bullet$ \textbf{Problem 2}
\medskip

\begin{itshape}
Is $T: \mathcal{P}_2(\mathbb{R}) \to \mathcal{P}_4(\mathbb{R})$, $T(p) = p(x^2+1)$ (i.e. $p$ composed with $x^2+1$) a linear transformation? Justify.
\end{itshape}
\medskip

\begin{proof}
$ $\newline
Claim: Yes

Let $p,q \in \mathcal{P}_2(\mathbb{R})$ and $\lambda \in \mathbb{R}$. Then
\begin{align*}
T(\lambda p) &= (\lambda p)(x^2+1) \\
&= \lambda \cdot p(x^2+1) \\
&= \lambda \cdot T(p(x)) 
\end{align*}
\begin{align*}
T(p+q) &= (p+q)(x^2+1) \\
&= p(x^2+1) + q(x^2+1) \\
&= T(p) + T(q)
\end{align*}
Hence $T$ is a linear transformation.
\end{proof}

\newpage
$ \bullet$ \textbf{Problem 3}
\medskip

\begin{itshape}
Let $T_1$ and $T_2$ be linear transformations from $\mathcal{V}$ to $\mathcal{W}$. Prove that $T_1 + T_2$ is a linear transformation.
\end{itshape}
\medskip

\begin{proof}
$ $\newline
\end{proof}

\newpage
$ \bullet$ \textbf{Problem 4}
\medskip

\begin{itshape}
Give an example of a linear transformation $T$ such that $dim \; N(T) = 3$ and  $dim \; R(T) =2$.
\end{itshape}
\medskip

\begin{proof}
$ $\newline
\end{proof}

\newpage
$ \bullet$ \textbf{Problem 5}
\medskip

\begin{itshape}
Prove there is no linear transformation $T: \mathbb{R}^5 \to \mathbb{R}^5$ with $R(T) = N(T)$.
\end{itshape}
\medskip

\begin{proof}
$ $\newline
\end{proof}

\newpage
$ \bullet$ \textbf{Problem 6}
\medskip

\begin{itshape}
Find a linear transformation $T: \mathbb{R}^4 \to \mathbb{R}^4$ such that $R(T)=N(T)$.
\end{itshape}
\medskip

\begin{proof}
$ $\newline
\end{proof}

\newpage
$ \bullet$ \textbf{Problem 7}
\medskip

\begin{itshape}
Let $\mathcal{V}$ be a vector space (not necessarily finite dimensional) and $\mathcal{S}_1 \subseteq \mathcal{S}_2 \subseteq \mathcal{V}$. Suppose $\mathcal{S}_1$ is linearly independent and $\mathcal{S}_2$ spans $\mathcal{V}$. Prove there is a basis $\mathcal{B}$ for $\mathcal{V}$ with $\mathcal{S}_1 \subseteq \mathcal{B} \subseteq \mathcal{S}_2$.
\end{itshape}
\medskip

\begin{proof}
$ $\newline
\end{proof}

\end{document}