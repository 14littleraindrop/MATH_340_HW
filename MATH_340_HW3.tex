\documentclass{article}

\usepackage[english]{babel}
\usepackage{fancyhdr}
\pagestyle{fancy}
\fancyhf{}
\lhead{WINTER 2021 --- MATH 340 HW3}
\rhead{Helen (Yeu) Chen}
\setlength{\parindent}{0cm}
\usepackage{makecell}
\usepackage{amsfonts}
\usepackage{longtable}
\usepackage{amsmath}
\usepackage{amssymb}
\usepackage{amsthm}
\usepackage{systeme}

\fancyfoot[C]{\thepage}

\begin{document}


$ \bullet$ \textbf{Problem 1}
\medskip

\begin{itshape}
Prove that the function $T: \mathcal{P}_2(\mathbb{R}) \to \mathcal{P}_4(\mathbb{R})$, $T(p(x)) = p(x) \cdot (x^2 +1)$ is a linear transformation. Describe $\mathcal{N}(T)$. Is $x^4$ in $\mathcal{R}(T)$?
\end{itshape}
\medskip

\begin{proof}
$ $\newline
First show that $T$ is a linear transformation.

Let $p(x), q(x) \in \mathcal{P}_2(\mathbb{R})$ and $\lambda \in \mathbb{R}$. Then
\begin{align*}
T(\lambda p(x)) &= \lambda p(x) \cdot (x^2+1) \\
&= \lambda (p(x) \cdot (x^2+1)) \\
&= \lambda T(p(x))
\end{align*}
\begin{align*}
T(p(x) + q(x)) &= (p(x) +q(x)) \cdot (x^2+1) \\
&= (p(x) \cdot (x^2+1)) + (q(x) \cdot (x^2+1)) \\
&= T(p(x)) + T(q(x))
\end{align*}
This shows that $T$ is a linear transformation.
\medskip

Claim: $\mathcal{N}(T) = \{ p(x) = 0 \}$

Write $p(x) \in \mathcal{P}_2(\mathbb{R})$ as $p(x) = ax^2+bx+c$. We want to fine all $p(x)$ such that $T(p(x)) = 0$.
\begin{align*}
T(p(x)) &= T(ax^2+bx+c) \\
&= (ax^2+bx+c) \cdot (x^2+1) \\
&= ax^4+bx^3+(a+c)x^2+bx+c
\end{align*}
So, $T(p(x)) = 0$ if and only if $a,b,c = 0$. Hence, only $p(x) = 0$ is in the null space of $T$.
\medskip

Claim: $x^4 \notin \mathcal{R}(T)$. If $x^4 \in \mathcal{R}(T)$, then there exists a $p(x) = ax^2+bx+c \in \mathcal{P}_2(\mathbb{R})$ such that $T(p(x)) =ax^4+bx^3+(a+c)x^2+bx+c= x^4$. However, the $x^4, x^3$ and constant components imply that $a=1, b=c=0$, but these coefficients will result in a nonzero $x^2$ component.  Hence, there is not $p(x)$ such that $T(p(x)) = x^4$, and so $x^4 \notin \mathcal{R}(T)$.
\end{proof}

\newpage
$ \bullet$ \textbf{Problem 2}
\medskip

\begin{itshape}
Is $T: \mathcal{P}_2(\mathbb{R}) \to \mathcal{P}_4(\mathbb{R})$, $T(p) = p(x^2+1)$ (i.e. $p$ composed with $x^2+1$) a linear transformation? Justify.
\end{itshape}
\medskip

\begin{proof}
$ $\newline
Claim: Yes

Let $p,q \in \mathcal{P}_2(\mathbb{R})$ and $\lambda \in \mathbb{R}$. Then
\begin{align*}
T(\lambda p) &= (\lambda p)(x^2+1) \\
&= \lambda \cdot p(x^2+1) \\
&= \lambda \cdot T(p(x)) 
\end{align*}
\begin{align*}
T(p+q) &= (p+q)(x^2+1) \\
&= p(x^2+1) + q(x^2+1) \\
&= T(p) + T(q)
\end{align*}
Hence $T$ is a linear transformation.
\end{proof}

\newpage
$ \bullet$ \textbf{Problem 3}
\medskip

\begin{itshape}
Let $T_1$ and $T_2$ be linear transformations from $\mathcal{V}$ to $\mathcal{W}$. Prove that $T_1 + T_2$ is a linear transformation.
\end{itshape}
\medskip

\begin{proof}
$ $\newline
Let $\vec{v_1}, \vec{v_2} \in \mathcal{V}$. Using the fact that $T_1$ and $T_2$ are linear transformations, we have
\begin{align*}
(T_1+T_2)(\lambda \vec{v_1}) &= T_1(\lambda \vec{v_1}) + T_2(\lambda \vec{v_1}) \\
&=\lambda T_1(\vec{v_1}) + \lambda T_2(\vec{v_1}) \\
&= \lambda (T_1(\vec{v_1})+ T_2(\vec{v_1})) \\
&= \lambda  (T_1+T_2)(\vec{v_1})
\end{align*}
\begin{align*}
(T_1 + T_2)(\vec{v_1} + \vec{v_2}) &= T_1(\vec{v_1} + \vec{v_2}) + T_2(\vec{v_1} + \vec{v_2}) \\
&= (T_1(\vec{v_1}) + T_1(\vec{v_2})) + (T_2(\vec{v_1}) + T_2(\vec{v_2})) \\
&= (T_1(\vec{v_1}) + T_2(\vec{v_1}) ) + (T_1(\vec{v_2}) + T_2(\vec{v_2})) \\ 
&= (T_1 +T_2)(\vec{v_1}) + (T_1 +T_2)(\vec{v_2})
\end{align*}
This shows that $T_1+T_2$ is a linear transformation.
\end{proof}

\newpage
$ \bullet$ \textbf{Problem 4}
\medskip

\begin{itshape}
Give an example of a linear transformation $T$ such that $dim \; \mathcal{N}(T) = 3$ and  $dim \; \mathcal{R}(T) =2$.
\end{itshape}
\medskip

\begin{proof}
$ $\newline
Let $T: \mathbb{R}^5 \to \mathbb{R}^5$, $T((x,y,z,t,w)) = (x,y,0,0,0)$ (i.e. the projection map).
\begin{align*}
\mathcal{N}(T) &= \{ (x,y,z,t,w) \in \mathbb{R}^5 | T((x,y,z,t,w))  = (0,0,0,0,0) \} \\
&= \{ (x,y,z,t,w) \in \mathbb{R}^5 | (x,y,0,0,0) = (0,0,0,0,0) \} \\ 
&= \{ (x,y,z,t,w) \in \mathbb{R}^5 | x = y =0 \}\\
&= \text{the }ztw \text{ plane}
\end{align*}
Clearly, the $ztw$ plane is just a three dimensional subset of $\mathbb{R}^5$ (namely $\mathbb{R}^3$). Hence, $dim \; \mathcal{N}(T) = 3$.

\begin{align*}
\mathcal{R}(T) &= \{ \vec{v} \in \mathbb{R}^5 | \vec{v} = T((x,y,z,t,w)) \text{ for some } (x,y,z,t,w) \in \mathbb{R}^5 \} \\
&= \{ \vec{v} \in \mathbb{R}^5 | \vec{v} = (x,y,0,0,0), x,y \in \mathbb{R} \} \\
&= \{ \text{the } xy \text{ plane} \}
\end{align*}
So clearly, $dim \; \mathcal{R}(T) = 2$.
\end{proof}

\newpage
$ \bullet$ \textbf{Problem 5}
\medskip

\begin{itshape}
Prove there is no linear transformation $T: \mathbb{R}^5 \to \mathbb{R}^5$ with $\mathcal{R}(T) = \mathcal{N}(T)$.
\end{itshape}
\medskip

\begin{proof}
$ $\newline
Assume there is such $T$ with $\mathcal{R}(T) = \mathcal{N}(T)$. Let $\text{nullity}(T) = \text{rank}(T) = n$, then we have a theory says that
\begin{align*}
dim(\mathbb{R}^5) &= \text{nullity}(T) + \text{rank}(T) \\
5 &= n + n \\
5 &= 2n \\
n &= 2.5
\end{align*}
However, we know that the dimension of a vector space must be an integer, hence $n \ne 2.5$, contradiction.

So, there is no linear transformation $T: \mathbb{R}^5 \to \mathbb{R}^5$ with $\mathcal{R}(T) = \mathcal{N}(T)$.
\end{proof}

\newpage
$ \bullet$ \textbf{Problem 6}
\medskip

\begin{itshape}
Find a linear transformation $T: \mathbb{R}^4 \to \mathbb{R}^4$ such that $\mathcal{R}(T)=\mathcal{N}(T)$.
\end{itshape}
\medskip

\begin{proof}
$ $\newline
Let $T((x,y,z,t)) = (z,t,0,0)$. First show that $T $ is a linear transformation.

Let $(x,y,z,t), (x',y',z',t')  \in \mathbb{R}^4$. Then,
\begin{align*}
T( \lambda (x,y,z,t)) & = T( (\lambda x, \lambda y, \lambda z ,\lambda t)) \\
&=  (\lambda z, \lambda t, 0 , 0) \\
& =  \lambda (z,t,0,0) \\
&= \lambda T((x,y,z,t))
\end{align*}
\begin{align*}
T((x,y,z,t) + (x',y',z',t')) &= T( (x+x', y+y', z+z', t+t')) \\
&= (z+z', t+t', 0,0) \\
&= (z,t,0,0) + (z',t',0,0) \\
&= T((x,y,z,t)) + T((x',y',z',t'))
\end{align*}
Hence, $T$ is a linear transformation.
\medskip

Now show that $\mathcal{R}(T) = \mathcal{N}(T)$.
\begin{align*}
\mathcal{N}(T) &= \{ (x,y,z,t) \in \mathbb{R}^4 | T((x,y,z,t)) = (0,0,0,0) \} \\
&= \{ (x,y,z,t) \in \mathbb{R}^4 | (z,t,0,0) = (0,0,0,0) \} \\
&= \{ (x,y,z,t) \in \mathbb{R}^4 | z=t=0 \} \\
&= \{ (x,y,0,0) | x,y \in \mathbb{R} \} \\
&= \mathbb{R}^2 \text{ (or the xy plane)}
\end{align*}
\begin{align*}
\mathcal{R}(T) &= \{ (x,y,z,t) \in \mathbb{R}^4 | (x,y,z,t) = T ((x',y',z',t')) \text{ for some } (x',y',z',t') \in \mathbb{R}^4 \} \\
&= \{ (x,y,z,t) \in \mathbb{R}^4 | (x,y,z,t) = (z',t', 0,0) \text{ for some } z',t' \in \mathbb{R} \} \\
&= \{ (x,y,0,0) | x,y \in \mathbb{R} \} \\
&= \mathbb{R}^2 \text{ (or the xy plane)}
\end{align*}
Hence, $\mathcal{R}(T) = \mathcal{N}(T)$.
\end{proof}

\newpage
$ \bullet$ \textbf{Problem 7}
\medskip

\begin{itshape}
Let $\mathcal{V}$ be a vector space (not necessarily finite dimensional) and $\mathcal{S}_1 \subseteq \mathcal{S}_2 \subseteq \mathcal{V}$. Suppose $\mathcal{S}_1$ is linearly independent and $\mathcal{S}_2$ spans $\mathcal{V}$. Prove there is a basis $\mathcal{B}$ for $\mathcal{V}$ with $\mathcal{S}_1 \subseteq \mathcal{B} \subseteq \mathcal{S}_2$.
\end{itshape}
\medskip

\begin{proof}
$ $\newline
Let $\mathcal{F}$ be a family of linearly independent subset of $\mathcal{S}_2$ that contains $\mathcal{S}_1$. 

That is,
$$\mathcal{F} = \{ \mathcal{A} \subseteq \mathcal{S}_2 | \mathcal{A} \text{ independent}, \mathcal{S}_1 \subseteq \mathcal{A} \} $$
Consider a chain in $\mathcal{F}$: $$\cdots \subseteq \mathcal{A}_1  \subseteq \mathcal{A}_2 \subseteq \mathcal{A}_3 \subseteq \cdots$$
Consider $\cup \mathcal{A}_i =\mathcal{C}$, then clearly $\mathcal{S}_1 \subseteq \mathcal{C} \subseteq \mathcal{S}_2$ since $\mathcal{S}_1 \subseteq \mathcal{A}_i \subseteq \mathcal{S}_2$ for all $\mathcal{A}_i$. Moreover, $\mathcal{C}$ is independent since if $\vec{v}_1, \vec{v}_2, \cdots, \vec{v}_k \in \mathcal{C}$ then there exist some $\mathcal{A}_t$ in the chain such that $\vec{v}_1, \vec{v}_2, \cdots, \vec{v}_k \in \mathcal{A}_t$. Since $\mathcal{A}_t$ is linearly independent, the set of vectors $\vec{v}_1, \vec{v}_2, \cdots, \vec{v}_k$ is also independent. Hence, $\mathcal{C}$ is independent. This shows $\mathcal{C} \in \mathcal{F}$ and it contains every set in the chains. The maximal principle then says $\mathcal{F}$ contains a maximal element $\mathcal{T}$. Notice that $\mathcal{S}_1 \subseteq \mathcal{T} \subseteq \mathcal{S}_2$ and $\mathcal{T}$ is independent.

I claim that $\mathcal{T}$ generates $\mathcal{S}_2$.
Suppose $Span(\mathcal{T}) \ne \mathcal{S}_2$, then there exists $\vec{v} \in \mathcal{S}_2$ with $\vec{v} \notin Span(\mathcal{T})$. So, $\mathcal{T} \cup \{ \vec{v} \} $ is linearly independent and clearly that $\mathcal{S}_1 \subseteq \mathcal{T} \cup \{ \vec{v} \} \subseteq \mathcal{S}_2$, this implies $\mathcal{T} \cup \{ \vec{v} \} \in \mathcal{F}$. However, this contradicts that $\mathcal{T}$ is a maximal element of $\mathcal{F}$. 

So, since $\mathcal{T}$ generates $\mathcal{S}_2$ and $\mathcal{S}_2$ spans $\mathcal{V}$, $\mathcal{T}$ generates $\mathcal{V}$. Moreover, since $\mathcal{T}$ is independent, it is a basis for $\mathcal{V}$. Thus, we have found our basis which is sandwiched between $\mathcal{S}_1$ and $\mathcal{S}_2$.
\end{proof}

\end{document}