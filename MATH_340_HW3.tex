\documentclass{article}

\usepackage[english]{babel}
\usepackage{fancyhdr}
\pagestyle{fancy}
\fancyhf{}
\lhead{WINTER 2021 --- MATH 340 HW2}
\rhead{Helen (Yeu) Chen}
\setlength{\parindent}{0cm}
\usepackage{makecell}
\usepackage{amsfonts}
\usepackage{longtable}
\usepackage{amsmath}
\usepackage{amssymb}
\usepackage{amsthm}
\usepackage{systeme}

\fancyfoot[C]{\thepage}

\begin{document}


$ \bullet$ \textbf{Problem 1}
\medskip

\begin{itshape}
Prove that the function $T: \mathcal{P}_2(\mathbb{R}) \to \mathcal{P}_4(\mathbb{R})$, $T(p(x)) = p(x) \cdot (x^2 +1)$ is a linear transformation. Describe $N(T)$. Is $x^4$ in $R(T)$?
\end{itshape}
\medskip

\begin{proof}
$ $\newline
\end{proof}

\newpage
$ \bullet$ \textbf{Problem 2}
\medskip

\begin{itshape}
Is $T: \mathcal{P}_2(\mathbb{R}) \to \mathcal{P}_4(\mathbb{R})$, $T(p) = p(x^2+1)$ (i.e. $p$ composed with $x^2+1$) a linear transformation? Justify.
\end{itshape}
\medskip

\begin{proof}
$ $\newline
\end{proof}

\newpage
$ \bullet$ \textbf{Problem 3}
\medskip

\begin{itshape}
Let $T_1$ and $T_2$ be linear transformations from $\mathcal{V}$ to $\mathcal{W}$. Prove that $T_1 + T_2$ is a linear transformation.
\end{itshape}
\medskip

\begin{proof}
$ $\newline
\end{proof}

\newpage
$ \bullet$ \textbf{Problem 4}
\medskip

\begin{itshape}
Give an example of a linear transformation $T$ such that $dim \; N(T) = 3$ and  $dim \; R(T) =2$.
\end{itshape}
\medskip

\begin{proof}
$ $\newline
\end{proof}

\newpage
$ \bullet$ \textbf{Problem 5}
\medskip

\begin{itshape}
Prove there is no linear transformation $T: \mathbb{R}^5 \to \mathbb{R}^5$ with $R(T) = N(T)$.
\end{itshape}
\medskip

\begin{proof}
$ $\newline
\end{proof}

\newpage
$ \bullet$ \textbf{Problem 6}
\medskip

\begin{itshape}
Find a linear transformation $T: \mathbb{R}^4 \to \mathbb{R}^4$ such that $R(T)=N(T)$.
\end{itshape}
\medskip

\begin{proof}
$ $\newline
\end{proof}

\newpage
$ \bullet$ \textbf{Problem 7}
\medskip

\begin{itshape}
Let $\mathcal{V}$ be a vector space (not necessarily finite dimensional) and $\mathcal{S}_1 \subseteq \mathcal{S}_2 \subseteq \mathcal{V}$. Suppose $\mathcal{S}_1$ is linearly independent and $\mathcal{S}_2$ spans $\mathcal{V}$. Prove there is a basis $\mathcal{B}$ for $\mathcal{V}$ with $\mathcal{S}_1 \subseteq \mathcal{B} \subseteq \mathcal{S}_2$.
\end{itshape}

\end{document}